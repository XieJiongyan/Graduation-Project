\chapter{数值计算方法}
为了实现对超空泡物体进入波浪场的模拟,本文借助流体力学计算软件 Fluent 对不同的入水条件进行数值模拟。本文首先确定了计算的几何模型和区域,进行恰当的网格划分,给予合适的 UDF 边界条件,选用适当的数值方法,从而实现了波浪场的模拟。之后借助界面捕捉的 VOF 方法以及重叠网格技术,实现了航行体入水的方案设计。
\section{几何模型和网格划分}
本文为了模拟五级海况\cite{Xia1987}的环境,选取的环境场背景网格为 $300 \mathrm m \times 60 \mathrm m \times 10 \mathrm m$ 的长方体区域。模拟水深 $40 \mathrm m$,波长 $40 \mathrm m$,波高近 $3 \mathrm m$。由于水面附近的计算精度要求高,故以水面所在平面为最精细的网格,两侧以一定比例按等比数列扩大其网格间距。入水过程发生在波浪船舶方向的 $0 < x < 200$ 范围内,后半区域 $200 < x < 300$ 为消波区。工作区划分为均匀网格,每个波长内大约80个网格,消波区采用快速增大的粗糙网格,总体网格量为84万,可以足够精细地描述波浪场环境。

运动网格为围绕入水细长航行体的圆柱体网格。细长航行体的长度$5 \mathrm m$,直径$0.5 \mathrm m$。网格沿圆柱体轴线方向采用等距划分,沿径向方向从外至内按照等比数列距离缩小,以增强靠近物体表面部分的精度。在模拟过程中,动网格始终环绕着入水物体,与入水物体之间的相对位置保持不变,并使用重叠网格技术以进行区域内的数值模拟。在初始状态下,背景网格与运动网格如图~\ref{fig:overlay_grid}所示

\begin{figure}[!htp]
  \centering
  \includegraphics[]{overlay_grid.png}
  \caption[]{入水角度 $45 ^ \circ$ 的初始重叠网格}
  \label{fig:overlay_grid}
\end{figure}

\section{边界条件}
采用速度边界作为输入的造波方法进行二阶 Stokes 波的模拟,入口处实时给定波高$\eta$以及 $x$ 方向和 $y$ 方向的水质点速度($u$、$v$)\cite{Zou2005}。
\begin{equation}
  \eta = \frac H 2 \cos (kx - \omega t + \Phi _0) + \frac {k H^2 \cosh k y} {16 \sinh ^3 kd} (\cosh 2kd + 2)cos2(kx - \omega t)
\end{equation}
\begin{equation}
  u = \frac {\pi H} T \frac {\cosh k(y + d)} {sinh kd} \cos (kx - \omega t) + \frac 3 4 \frac {\pi H} T \left( \frac {\pi H} L \right) \frac {\cosh 2k(y + d)} {\sinh^4 kd} \cos 2(kx - \omega t)
\end{equation}
\begin{equation}
  v = \frac {\pi H} T \frac {\sinh k(y + d)} {sinh kd} + \frac 3 4 \frac {\pi H} T \left( \frac {\pi H} L \right) \frac {\sinh 2k(y + d)} {\sinh^4 kd} \sin 2(kx - \omega t)
\end{equation}

在确定了左侧入口处的速度边界条件后,为了保证入口和出口流量守恒,右侧出口给定与入口 Stokes 波的静漂移相对应的均匀速度。本模型模拟的是三维环境,但 z 方向的动力学特性是轴对称的,因而 z 方向的两个表面采用的是对称面边界条件。$y = 0$ 的边界为海底,给定固壁边界条件,$y = 60$ 的边界为大气,给定压力出口边界。

\section{动量源项消波法}
单纯采用速度边界作为输入时,会发生波浪反射的现象,影响波浪场效果。对于这个问题,可以采用动量源项消波法使结果更准确\cite{Wang2005,Li2013,Peric2015}。本文使用动量源项消波法的区域为 $200 < x < 300$ 的消波区。对于 Navier-Stokes 方程,增加一动量源项,有
\begin{equation}
  \frac {\partial (\rho u_i)}{\partial t} + \frac {\partial (\rho u_i u_j)}{\partial x_j} = \frac {\partial}{\partial x_j} \left[ \mu \left( \frac {\partial u_i}{\partial x_j} + \frac {\partial u_j}{\partial x_i} \right) \right] - \frac {\partial}{\partial x_i} \left( p + \frac 2 3 \mu \frac {\partial u_k}{\partial x_k} \right) + S_i
\end{equation}
其中 $\mathbf S$ 即为增加的动量源项,可采用如下形式确定:
\begin{equation}
  S_i = [C(x) - 1]\left[ \frac \rho {\Delta t} (u_{iC} - U_i) - \rho u_{jC} \frac {\partial u_{iC}}{\partial x_j} - \frac {\partial p_c} {\partial x_i} + \rho f_i \right]
\end{equation}
其中 $C(x)$ 为任意光滑函数,下标 $C$ 为计算值。在本次模拟场景下,C(x)取值为
\begin{equation}
  C(x) = 0.9985 + 0.0015 t - 0.9985 e^{-t/0.025}, t = 1 - \frac x l
\end{equation}
其中 $x$ 为质点距消波区左端的距离, $l$ 为消波区长度

\section{数值方法}
数值模拟在上述网格划分和边界条件等的基础之上,借助商业计算流体软件 Fluent,设定具体地数值方法。模型整体是基于压力的时变迭代方法,考虑重力的影响。以水为主相,气为次相,水相为不可压缩流体,气相采用理想气体模型考虑其压缩性。采用RNG k-epsilon模型结合壁面函数的方法计算湍流效应。数值算法采用压力速度耦合算法。

\section{界面捕捉的 VOF 方法}
在通常的数值模拟过程中,我们在每个单元中对每相使用一个值 $\alpha$ 来表示该相的体积分数,并确保各项的体积分数之和为 $1$,但这种表示方法没有给出各项在这个单元内的交界面形状,使用 VOF 方法则可以重构出交界面形状\cite{HIRT1981},例如,使用平行于网格的直线对交界面进行拟合,对应的则为 SOLA-VOF 算法。本模拟过程采用的是二阶 PLIC 算法,可以得到更精确的自由面和空泡形状。这种 VOF 界面捕捉方法与前述的动量源项消波法完全兼容\cite{Li2013}。

\section{重叠网格技术}

% --- 这块是复制粘贴的

我们将航行体包裹在一小块子区域中,该子区域处于整个计算区域工作区的适当位置。针对子区域和外部计算域的不同疏密要求分别划分网格。子区域与外部区域的交界处通过滑移面相互匹配,并使网格尽量均匀过渡。
由于入水问题的计算域是随时间变化的,我们将采用动网格技术实现不断变化的求解域。动网格的方法根据每个时刻边界的新位置对网格进行自动更新。存在移动边界的任意控制体上的某一标量 $\phi$ 的守恒方程的积分形式可以写作:

\begin{equation}
  \int _V \frac {\partial (\rho \phi)} {\partial t} \mathrm dV + \int_A \rho \phi (u_i - (u_g)_i) \mathrm d A_i = \int_V S_\phi \mathrm dV + \int _A \Gamma \frac {\partial \phi}{\partial x_i} \mathrm d A_i
\end{equation}
其中 $u_g$ 是网格移动速度。

边界的运动导致计算域的网格也要发生相应的运动和变形。动网格方法中,有一类区域的网格只发生一定的运动,构成单元的各个面的运动速度相同,因此网格本身并没有变形。此时,只要采用如上式的控制方程即可,而不必额外对网格进行调整。另一类区域的网格必须通过变形才能调整,它根据当前时刻的边界位置和速度以及时间步长,确定下一时刻的边界位置,再在邻近移动边界的局部区域对网格进行调整,甚至重新划分网格。弹簧平滑方法、动态分层方法、网格重构等方法都是较为常用的方法。针对航行体出水问题的特点,我们在计算过程中将尽量采用动态分层方法,通过滑移面使包裹着航行体的子区域相对于外部区域向上运动。

在结构化网格中,所有与移动区域相邻的网格拓扑结构都是柱形,可以使用动态铺层方法。它根据动边界附近层的高度来添加或移除与移动边界相邻的网格层。

动态铺层法指定在每一个动边界附近的理想层高度。根据网格层$j$的高度,相邻于动边界的网格层$j$被分割或者与附近的网格层$i$合并。如果$j$层上的网格拉伸了,网格高度可以扩展到:

\begin{equation}
  h_{\min} > (1 + \alpha _s) h_{\mathrm{ideal}}
\end{equation}

式中, $h_{\min}$ 是$j$层网格的最小高度,$h_{\mathrm{ideal}}$ 是理想网格高度, $\alpha _s$是分裂因子。当达到上述条件后,网格将根据指定的固定高度或者固定比率进行分割。当指定固定高度时,网格层会分裂成两层,一层是固定高度$h_{\mathrm{ideal}}$,另一层是$h - h_{\mathrm{ideal}}$ ;当指定固定比率时,新生成的网格高度比率为 $\alpha_s$,如果$j$层上的网格被压缩时,压缩将进行到下式的程度:
\begin{equation}
  h_{\min} < \alpha _c h_{\mathrm {ideal}}
\end{equation}
式中, $\alpha _c$是溃灭因子。当达到上面的条件时,被压缩的层$j$就会和它之上的层$i$合并。

对于非结构化网格,采用局部网格重构法更新网格。
局部网格重构法将那些不满足倾斜度或者大小准则的网格标记,然后重新生成网格,同时将计算得到的变量值插值到旧网格上:(a)网格大于指定的最大网格尺寸。(b)网格小于指定的最小网格尺寸。(c)网格倾斜度大于指定的最大倾斜度。
以简单的四面体网格构成的圆柱为例。当其底部移动时,采用与动态铺层法类似的方法,分析连接在边界上的网格高度,根据指定的理想面高和分离/合并因子来分离或合并网格。当第$j$层网格拉伸时,如果$h_{\min} > (1 + \alpha_h) h_{\mathrm{ideal}}$,网格面就根据预先设定的面高分裂,新的面高等于理想面高。当第$j$层压缩时,如果满足$h_{\min} < \alpha _h h_{\mathrm{ideal}}$,被压缩的层$j$则与之上的层$i$合并。这里的$h_{\mathrm{ideal}}$是理想面高,$\alpha$是高度因子。

\section{实验设计}

为了探究不同入射角度和入射相位对航行体运动的影响,在方案设计中会针对不同入射角度和入射相位进行数值模拟实验。在对航行体入水过程进行模拟之前,首先使航行体在水面上方以一定的姿态保持静止状态,以选取的波浪参数进行数值水池造波。然后,待工作区的波面形状趋于稳定状态之后,采用重叠网格方法和动网格技术使航行体的网格子区域开始向下(垂直或带倾角)运动。根据所需要的入水波浪相位,调整好恰当的航行体启动时刻,以指定的入水速度和入水角度运动至水面,之后的入水阶段采用完全自由运动。

针对航行体在不同波浪相位($0 ^\circ$ - $270 ^\circ$)入水过程进行数值模拟时,$0^\circ$相位表示航行体头部触及水面时位于波浪的波峰位置,$180^\circ$相位表示位于波谷位置,$90^\circ$和$270^\circ$相位表示位于波峰和波谷的中间位置。

对于不同的波浪相位和航行体倾角,航行体在启动时刻之后的初始运动速度均统一为 $10 m/s$,运动总时长 $t$ 为 $2s$,每个时间步为 $0.001s$,每个时间步迭代 50 次。