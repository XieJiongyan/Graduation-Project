\chapter{超空泡入水问题的基本理论}
超空泡入水过程是一个多介质且时变的流体问题,其基本理论可由多相流数学模型和入水空泡模型进行阐述。
\section{多相流数学模型}
在航行体超空泡进入波浪场的环境下,整体环境为气-液-汽三相的波浪场环境,空气与水均满足流体基本的连续性方程,动量方程以及能量守恒方程,
\begin{equation}
  % \frac{\partial \rho}{\partial t} + \frac{\partial}{\partial x_i} \(\rho u_i\) = 0
  \frac {\partial \rho}{\partial t} + \frac {\partial}{\partial x_i} (\rho u_i) = 0
\end{equation}
\begin{equation}
  \frac {\partial (\rho u_i)}{\partial t} + \frac {\partial (\rho u_i u_j)}{\partial x_j} = \frac {\partial}{\partial x_j} \left[ \mu \left( \frac {\partial u_i}{\partial x_j} + \frac {\partial u_j}{\partial x_i} \right) \right] - \frac {\partial}{\partial x_i} \left( p + \frac 2 3 \mu \frac {\partial u_k}{\partial x_k} \right)
\end{equation}
\begin{equation}
  \frac \partial {\partial t} (\rho E) + \frac \partial {\partial x_i} \left( u_i (\rho E + p) \right) = \frac \partial {\partial x_i} \left( k \frac {\partial T}{\partial x_i} \right)
\end{equation}

上式为连续性方程,动量方程以及能量守恒方程的张量表示法,其中$\rho$为流体密度,$\mu$ 为动力学粘性系数,$p$ 为压强,$E$ 为内能与动能之和。$u_i$为流体速度,是一个三维向量。

实际的入水过程都需要借助湍流模型表述\cite{Kliafas1987},对于流场中的气体和液体,均满足
\begin{equation}
  \frac {\partial \rho \phi}{\partial t} + \nabla \cdot (\rho \phi \mathbf u) = \nabla \cdot (\Gamma \nabla \phi) + S_{\phi}
\end{equation}
其中 $\phi$ 是流场中关于 $\mathbf x$ 和 $t$ 的连续函数。

在气液两相流的环境下,需要引入各相的体积分数,满足,
\begin{equation}
  \sum _{q = 1} ^3 \alpha _q = 1
\end{equation}
由于两相流模型中气液均满足相同形式的控制方程,因而混合介质下的各物理量可表达为以下形式:
\begin{equation}
  \phi = \begin{cases}
    \sum _{q = 1} ^3 \alpha_q \phi_q & \phi = \rho, \mu \\
    \sum _{q = 1} ^3 \alpha_q \phi_q \rho_q \bigg/ \sum _{q = 1} ^3 \alpha _q \rho _q & \phi = E, T
  \end{cases}
\end{equation}
而各相还需要满足输运方程:
\begin{equation}
  \frac {\partial (\rho _q \alpha _q)} {\partial t} + \frac {\partial (\rho _q \alpha _q u_i)}{\partial t} = \begin{cases}
    0 & q = gas \\
    S & q = vapor
  \end{cases}
  \label{eq:transport}
\end{equation}

\section{入水空泡模型}

\subsection{自由面空气和气体空泡模型}
本项目的入水问题采用 Mixture 模型,对于航行体入水时形成的空泡,其中气相体积也满足方程\ref{eq:transport},故有,
\begin{equation}
  \frac \partial {\partial t} (\rho _g \alpha_g) + \frac \partial {\partial x_i} (\rho _g \alpha _g u_i) = 0
\end{equation}

\subsection{空化模型}
当物体超高速($10^3 m/s$)入水时,会发生显著的自然空化现象,其中蒸汽相满足以下形式的输运方程:
\begin{equation}
  \frac {\partial \alpha _v}{\partial t} + \frac \partial {\partial x_j} (u_j a_v) = S_e - S_c
\end{equation}
其中 $S_e - S_c$ 为相变速率,典型的空化模型中,有,
\begin{equation}
  S_e = C_e \rho _v \alpha _l \frac {\max (p_v - p, 0)} {(1 / 2 p_l U _{\infty} ^2) t _{\infty}}
\end{equation}
\begin{equation}
  S_c = C_c \frac {\rho _v}{t _{\infty}} (\alpha _l - \alpha _g) ^2 (1 - \alpha_l 0 \alpha_g)
\end{equation}
其中 $C$ 为经验系数,下标 $l$、$v$、$g$ 分别代表液相、蒸汽相和气相,$\alpha$ 为各相的体积分数。