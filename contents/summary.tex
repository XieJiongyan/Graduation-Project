% !TEX root = ../main.tex

\begin{summary}
  入水问题是当前科研领域的一个关键问题,研究由来已久。当前阶段,空泡入水是入水问题研究的热点方向。空泡入水问题在波浪场环境下的研究相对不够充分,因此本论文研究了波浪场环境下的空泡入水问题。

  本论文首先对入水空泡这一问题的国内外研究背景进行了详细论述。早在 20 世纪初期,入水时受到的冲击就已经有了许多研究,随后至 20 世纪后半页,入水空泡和空化问题逐渐引起了研究的重视。对于入水问题,早期的研究方法包括理论求解,实验验证等,后来逐步开始采用有限元,PIV 或者 CFD 数值模拟的方法。本论文使用的是 CFD 方法对入水空泡问题进行数值模拟。

  在本论文的第二部分,介绍了航行体进入波浪场的部分原理。包括控制流体运动的基本原理,包括连续性方程,动量方程以及能量守恒方程。之后,又论述了多相流环境下的流体力学基本方程的相关变化。最后又论述了入水空泡模型。这些数学模型是进行数值模拟的基础。

  在本文的第三部分,介绍了本实验使用的数值模型。本实验借助流体力学计算软件 Fluent 进行数值模拟。首先,本文介绍了采用的几何模型和网格划分,其次,给定了各个边的边界条件,并论述了具体的消波方法。为了防止出水区边界造成波浪反射现象,在波浪场中增加一部分消波区,并且在该区域内采用动量源项消波法的方法。为了更好地捕捉界面,采用 VOF 方法进行捕捉。航行体入水过程中,为了让航行体周围的流场情况更加精确,采用重叠网格技术,将包裹航行体的动网格和环境的静态网格可以融合在一起进行流场求算。最后论述了本次实验的具体设置实验设计,根据不同入射角度和入射相位,总计进行了 9 组实验。

  本文的第四部分则对 9 组数值模拟实验进行分析和现象归纳。首先展示了稳定的波浪场结果,其次大致描述了不同入水阶段的密度分布。之后以 $60 ^\circ$ 入水和 $90 ^\circ$ 入水的各一个算例为例,详述了入水过程中的密度场,压力场和速度场。最后分析和论述了各算例下航行体入水过程的特点。
\end{summary}
