\chapter{研究背景}

随着全球各军事强国科技实力的不断增强,海洋日趋成为各国竞相争夺的现代战争主战场。作为现代海战武器装备的重要组成部分,空投鱼雷、反潜导弹、火箭深弹、高速射弹等水下武器的发展一直是各国非常重视的热点问题。自上世纪后期以来,俄罗斯“暴风雪”超空泡鱼雷和美国“RAMICS”超空泡射弹系统的出现更是将国际尖端水下武器的研制带入了新的研究热潮。我国要在未来成为海上军事强国就必须重视这些先进水下航行体的水动力特性和机理的研究。

水下航行体从机载或舰载平台发射后,必须经历穿越自由水面的入水过程。入水过程作为水下航行段的开始,是限制超空泡武器研发的一个关键瓶颈。首先,为了便于水下超空泡的形成宜采用高速入水,而高速入水冲击会造成航行体结构和器件的破坏;其次,入水空泡对入水后航行体的弹道稳定性产生很大影响。总之,入水过程是整个航行过程中流动现象和载荷最复杂、航行体姿态和弹道最不确定的非受控阶段,同时也为紧接着的水下航行阶段提供初始条件,其对于水下航行的可控性和末段攻击的有效性具有至关重要的作用。因此,入水过程是发射过程中最重要的环节,也是决定发射成败的关键所在。

\section{波浪场入水过程的物理现象}

入水过程涉及航行体和空气-水表面,在航行体的运动过程中,空气和水的流体特性占据主要影响。入水过程会发生航行体与空气-水表面的接触,因而会发生受力,受力矩等的突变,再考虑到流体本身的湍流特性和复杂性,整体是高度非线性问题。在波浪场中进行运动,波浪场本身也是一个时变系统,入水过程的具体动力学特性也与航行体入水时波浪场所处的相位息息相关。入水过程会发成空泡现象\cite{young1999cavitation,truscott2014water},具体地,包括航行体携空气入水和空化现象。当航行体运动速度较低时,航行体会携带一部分空气进入水中,形成空泡;当航行体运行速度更高时,在普通空泡的基础上,还会发生空化现象:由于运动速度很高,使得局部压力变得很低,使得水发生了蒸发,这些水蒸气形成了空化的气泡。

航行体入水的科学问题涉及介质密度突变、瞬间极端载荷、流体压缩性、移动接触线、自由面大变形、入水空泡演化、多相间作用、湍流流动等诸多物理效应,具有多因素耦合作用的高度非线性、非稳态和多尺度的特点。由于从空气到水的介质密度突跃,航行体撞击水面时瞬间产生巨大的砰击载荷,严重威胁航行体的结构安全性和入水弹道稳定性,特别是斜向入水时易发生跳弹、忽扑、甚至沉底等严重后果。高速入水瞬间还会产生向水体中辐射的冲击波,须考虑液体压缩性的影响。由于气液动态接触角所代表的航行体表面亲疏水性(沾湿性)的不同,入水现象可分为无空泡入水和带空泡入水。接触角的大小决定了能够产生入水空泡的临界毛细管数Ca(或临界速度),而航行体入水速度一般远高于其临界速度,因而大多表现为带空泡入水。入水后的头部沾湿面积增长至一个稳定的空泡分离线为止,而航行体构型和入水条件等则共同影响着变化多端的入水空泡形态。入水空泡根据空泡生成物理机制的不同又可分为两类\cite{truscott2014water},如图\ref{fig:background}所示:(I)一般高速入水时,空泡膨胀抽吸水面空气形成高速气流,泡内气流压力低于大气压,形成携带空气的气体空泡。(II)超高速入水($O(10^3) m/s$)时,泡内空气远远偏离平衡态而造成极低压,自然空化效应形成含水蒸汽的两相超空泡。航行体带空泡入水过程虽然短暂,但可细分为水面抨击、流动形成、空泡敞开发展、空泡闭合、空泡溃灭、全湿航行等几个阶段,图2以圆球入水为例给出主要过程示意图。抨击作用将动量传输给水体并建立起扰动流场,同时航行体携带空气形成敞开发展的空泡。在周围水体的约束作用下,空泡膨胀迫使流动向无约束的自由水面发展,形成自由面隆起,继而产生分离液鞘和飞溅液滴。航行体在泡内运行的尾拍和滑行等模式还会与空泡相互作用形成二次干扰泡。空泡膨胀吸入的高速气流(甚至在空泡收缩喉部达到超声速\cite{gekle2010supersonic})造成泡内压力降,随着空泡径向惯性动能的持续消耗、以及泡内与周围环境压差的不断增大,空泡在自由面附近或中段发生闭合和断裂。空泡闭合现象存在表面闭合、深闭合、浅闭合、准静态闭合等多种形式。空泡闭合瞬间产生两股反向高速射流,射流冲击空泡壁面和航行体尾部,破坏空泡形态及造成弹道失稳。当首次闭合后的空泡较长时还会发生多次闭合脱落现象,不断干扰航行体的运动。空泡完全闭合后,航行体带局部附着空泡航行,局部空泡会由于水流的挟带作用而发生的非稳态脱落、突发不对称滑脱等现象,引起航行体运动方向的不稳定。

\begin{figure}[!htp]
  \centering 
  \includegraphics[width = 0.9\textwidth]{backgroud_1.png}
  \caption{航行体入水空泡,(a):细长航行体入水空泡;(b):高速入水的多相超空泡}
  \label{fig:background}
\end{figure}

\section{航行体入水过程的研究方法}

入水过程是一个非常复杂的非线性问题,尽管我们已经有了非常丰富的对流体各问题的数学描述,但想要单纯通过理论对具体的入水过程进行求解接近于不可能。早期对于入水问题的研究的一个重要问题是想要通过理论分析计算出入水过程中物体受到的最大冲击力,但被证明是非常困难的。尽管实验是研究该过程的最可靠方法,但在实践中,通过实验评估航行体上的水压是不现实的,因为这需要安装大量压力传感器来覆盖所有敏感的航行体上的区域。因此,测量只能在有限的位置进行,这意味着评估不会是全面的。或者,建模方法可以克服传感器数量的限制,但很难准确预测航行体上的水致压力。以有限元分析法 (FEA) 为例,它需要在航行体表面的特定位置输入压力来进行相关的结构分析,但 FEA 本身并没有可靠的来源来提供该压力输入。因此,之前对入水航行体的有限元分析研究是基于准静态假设建立的,其中水压可以从实验中获得(同样是有限的位置),或者根据重力加速度和动压公式估算,即$H = 1 / 2 gt ^2$, $v = gt$, $P = 12 \rho_{water}v^2$,其中$H$是航行体的下落高度,$P$是可以标定并输入FEA的水压。因此,不可靠的水压源在以下有限元分析结构评估中增加了相当大的不确定性。进水问题的另一种方法是面板积分法,它依赖于势流理论,其中离散化方法用于将物体划分为有限数量的表面,然后可以获得 Froude-Krylov 力。 Zhao 和 Faltinsen \cite{zhao1993water}使用这种方法进行了具有代表性的进水研究,该方法至今仍被广泛用作基准。尽管如此,他们的解决方案是针对零重力条件提出的,这意味着它可能仅与允许忽略重力的进水过程物理匹配。然后将该方法扩展到包括重力\cite{sun2007influence}和非线性速度属性\cite{wu2010summary}。然而,这种类型的势流解决方案不能解释粘性和湍流的流体行为,仍然限制了它的工程适用性。

最广泛的航行体入水的研究方法是使用基于网格的 CFD 方法,或称为有限体积法,这种方法应用广泛且成熟。CFD 已广泛应用于预测流体行为\cite{pena2019numerical}和流体引起的结构载荷、运动和变形\cite{huang2019fluid, dashtimanesh2020numerical, tavakoli2021wave}。据研究,CFD 的准确性对于固体与包含自由表面的多相流相互作用的流体动力学问题非常好\cite{windt2020validation, javanmard2020new, huang2021simulation},对粘性和湍流进行了很好的建模\cite{khojasteh2020numerical}。本文使用成熟的商业软件 Fluent 进行航行体入水过程的模拟。

\section{国内外发展现状}

纵观物体入水问题的研究历史,其研究方向主要沿着入水冲击和入水空泡这两条主线展开,彼此之间是相互联系和交织的,两者又各有试验研究、理论建模和数值计算等几种研究手段。从国内外的研究历程来看,大体可分为早期(19世纪末~20世纪初)、中期(第二次世界大战后至20世纪末)和近代(20世纪90年代迄今)三个发展阶段。

关于入水冲击问题,最大冲击载荷的理论预报和试验测量都是非常困难的,理论难点在于必须考虑自由面的非线性变形,试验难点则在于试验装置难以承受巨大冲击载荷和实现快速动态响应。入水冲击问题最早的先驱性理论工作是Von Karman\cite{von1929impact}提出的附加质量方法,其后国际上的理论研究大多都是该理论的延伸。Wagner\cite{wagner1932phenomena}在此基础上针对小斜升角二维楔形体的垂直入水问题提出了近似平板理论,为入水冲击的研究奠定了重要的理论基础,后人的很多理论方法都是Wagner问题的改进和拓展。入水冲击的最早试验研究则是由Watanabe开展的圆锥、圆球和圆板等的入水试验,为后人理论模型的发展提供了关键的验证数据。迄今为止,国内外在这些先驱工作的基础之上围绕入水冲击开展了大量研究探索\cite{sedov1934impact, yu1945virtual, shiffman1945force, courant1945force, shiffman1951force, hillman1946vertical, pierson1950penetration, monaghan1949theoretical, bisplinghoff1952some, fabula1957ellipse, trilling1950impact, schnitzer1953estimation, faltinsen1998water, cointe1989two, miloh1991oblique, miloh1991initial, howison1991incompressible, zhao1993water, lin1994water, anghileri1995experimental, mei1999water, park2003numerical, chen1990, gu1991, 郑际嘉1992刚性圆板自由落体在水面上的冲击压力, 钱勤1994任意的拉格朗日欧拉边界元, lu1998nonlinear, 张军2003楔形体入水初期流场的数值模拟, 张于维2010二维楔形体砰击载荷研究}。鉴于本项目的主要内容在于入水空泡,因此入水冲击的研究现状和发展动态这里不多赘述。

物体入水空泡的研究最早始于19世纪末Worthington的开创性工作。他运用闪光摄影技术研究了球体垂直入水过程,首次系统阐述了水面喷溅、空泡闭合和射流等流动现象。尽管受当时试验条件所限,他的研究思路和关于入水现象的定义仍然被后来的研究者沿用,如Bell和Maccoll\cite{maccoll1928aerodynamics}分别开展了一系列圆球垂直入水试验,对入水空泡的形成机理和发展过程给出了更为明确的物理描述。这些研究只是出于对物理现象本身的兴趣,在当时并没有明确的应用背景。20世纪第二次世界大战后,由于水下武器发展的迫切需要,人们开始重新开展入水空泡的基础研究。Gilbarg和Anderson\cite{gilbarg1948influence}针对入水速度、大气压力和密度对入水空泡发展规律的影响开展了大量试验研究。May和Woodhull\cite{may1948drag}利用高速相机拍摄了球体入水过程,计算并建立了入水阻力与弗劳德数和雷诺数的函数关系。Richardson等\cite{richardson1948impact}研究了流体密度和粘度对球体垂直入水空泡演化的影响,同时结合势流理论对空泡发展过程进行了理论描述,并分析了球体倾斜入水的水面弹跳现象的机理。May和Woodhull\cite{may1950virtual}选取不同材质的球体研究了球体与水的相对密度、表面材料性质等对入水附加质量的影响。Birkhoff等对物体入水流场运动规律、入水空泡瞬态特性、空泡闭合射流以及泡内压强等进行了试验研究。Abelson\cite{abelson1970pressure}开展了射弹垂直入水空泡内部压力分布的试验研究。结果表明:空泡发生表面闭合之前的泡内压力降低量远大于预期,并且压降随入水角的减小而迅速减小。Logvinovich\cite{logvinovich1972hydrodynamics}基于能量守恒提出的空泡截面“独立膨胀原理”是最早描述空泡发展过程的理论,一直是入水空泡壁面运动理论研究的重要思想之一。May\cite{may1970review, may1975water}通过总结美国海军军械实验室长期的入水空泡试验结果,发展出了预测垂直入水空泡轮廓的“理想空泡模型”。总体来说,入水空泡的瞬态特点对于试验设备和试验技术的要求非常高,因此这一时期国内外还只有少数研究机构在国家军事需求的支撑下开展入水空泡研究。

进入21世纪之后,流体力学测试技术的进步大大促进了入水空泡研究的活跃度,大量研究成果在流体力学国际顶级学术刊物上持续发表。哈佛大学的Glasheen和Macmahon\cite{glasheen1996hydrodynamic}在研究美洲蜥蜴的水面快速行走时发现其脚底打击水面产生入水空泡,并将该过程简化为圆盘入水问题开展试验研究\cite{glasheen1996vertical},分析了低雷诺数下圆盘匀速垂直入水产生的空泡及其阻力变化。Gaudet\cite{gaudet1998numerical}则在低弗劳德数下数值模拟了圆盘垂直入水空泡和阻力系数的发展过程,发现了空泡深闭合位置与弗劳德数的对应相关。Gekle等\cite{gekle2008noncontinuous}则针对圆柱入水空泡总结了了空泡深闭合与弗劳德数之间的关系,表明在不同速度范围内出现不同的变化趋势。Lee等\cite{lee1997cavity}运用能量守恒原理建立了高速入水空泡生成和溃灭过程的空泡动力学理论模型,研究了高速入水空泡的生成、发展以及深闭合和表面闭合的发生过程与影响因素,给出了发生这两种空泡闭合的弗劳德数范围($20<Fr^2<70$和$Fr^2>150$)。Gordillo\cite{gordillo2005axisymmetric}等运用势流理论研究了高雷诺数下入水空泡最小半径及闭合点空泡形态随时间的变化关系,选择不同变化关系时可得到不同的闭合点形状。Oger等\cite{oger2006two}采用SPH方法计算了楔形体入水过程中的流固相互作用,得到了与试验吻合良好的压力场和物体运动特性。Duclaux等\cite{duclaux2007dynamics}在高雷诺数和高韦伯数下试验研究了圆球和圆柱入水空泡,总结了不同入水速度和物体直径条件下的深闭合位置和时间的规律,又采用势流理论并借鉴经典Besant-Rayleigh空泡溃灭问题,建立了泡径随时间变化的数学模型。Duez等\cite{duez2007making}以及Snoeijer和Andreotti\cite{snoeijer2013moving}关于表面沾湿性对气液分离影响的研究表明,对于任何气液接触角都存在一个生成入水空泡的临界毛细管数$Ca$($Ca>0.1$时任意接触角均能产生入水空泡)。Aristoff等\cite{aristoff2008water}试验研究了低邦德数下的入水空泡,指出空泡外形取决于表面张力与惯性力之比。Aristoff和Bush\cite{aristoff2009water}总结了入水空泡闭合的四种类型(深闭合、表面闭合、浅闭合和准静态闭合)的形态特征和发生条件。Antkowiak等\cite{antkowiak2007short}则详细研究了空泡闭合过程中产生的两股反向射流(包括著名的Worthington射流)的形成过程以及形状和大小。麻省理工学院的Truscott在Techet指导完成的博士论文\cite{truscott2009cavity}以及系列研究论文中,系统地开展了圆球垂直入水\cite{aristoff2010water,truscott2012unsteady}、亲疏水性圆球旋转入水\cite{truscott2009spin, truscott2009water, techet2011water}和高弹性圆球倾斜入水\cite{truscott2012holy}等方面的试验研究,利用粒子图像测速仪(PIV)测量了入水空泡流场,详细讨论了空泡形态演化及空泡对流体动力的影响。Truscott等\cite{truscott2009shallow}还开展了细长圆柱弹体高速倾斜入水的试验研究,分析了不同头型和材质弹体的入水弹道稳定性(包括跳弹现象)。Backer等\cite{de2009experimental}试验研究了圆锥体垂直入水的流动分离和空泡演化过程,讨论了物体入水后的速度、加速度和阻力系数的变化过程。Gekle等\cite{gekle2010supersonic}在圆盘入水空泡试验中发现,空泡颈缩闭合时的泡内气流速度随着空泡截面的缩小而迅速增加,喉部速度甚至达到超声速状态。Grumstrup等在物体高速入水试验中观察到闭合后空泡壁面的规则波动现象,通过测量空泡附近流场中的瞬间声信号变化,证实这种空泡脉动是由深闭合夹断瞬间的压力扰动所引起的声共振。Bodily等\cite{bodily2014water}在试验中采用物体内嵌加速度仪的方法,研究了各种头型和表面亲疏水性的细长航行体以不同倾角入水的过程,通过加速度积分计算和分析了水中运动阻力以及轨迹、速度、加速度和偏转角度与各种入水参数之间的关系。试验还发现,光滑头型的入水空泡大小不足以包裹细长航行体时,空泡深闭合会首先发生在航行体上。Enriquez等\cite{enriquez2012collapse}的试验也显示了航行体外形对空泡闭合行为的改变。他们开展了各种周向开槽圆盘的入水空泡扰动试验,发现形成菠萝状的非轴对称入水空泡,激发空泡面的高阶谐波振动,进而引发非常规的复杂深闭合现象。

国内学者长期以来关于入水空泡问题也开展了大量的研究工作。张庆明和陈九锡\cite{张庆明1984回转体垂直入水早期空泡的一个计算方法}采用高速相机开展了半球头圆柱体垂直入水空泡试验,分析了入水深度、速度和加速度随时间的变化过程。陈九锡和颜开\cite{陈九锡1986用}采用MAC方法计算了不同头型航行体的垂直入水过程,并分析了空泡从产生到溃灭的过程。叶取源\cite{叶取源1989锥头物体垂直入水空泡的发展和闭合, 叶取源1990用}基于非线性自由面理论,采用欧拉-拉格朗日混合边界元法和时间步进法,对锥头和平头航行体入水空泡的发展过程及表面闭合、深闭合等流动现象进行了数值计算,分析了弗劳德数和航行体质量对空泡演化的影响。李森虎、何友声和鲁传敬\cite{李森虎1992超声速平头物体垂直撞水的数值模拟}采用质点网格法(PIC)开展了超音速平头航行体垂直入水全过程的数值模拟,直观清晰地展现了入水气垫效应、水中激波传播以及入水空泡和自由面演化过程。进入21世纪后,国内的入水空泡研究也在向高精度、高分辨率试验测量和流场模拟的精细化方向发展。施红辉等\cite{shi2000optical}采用高速摄像手段研究了射弹以300m/s以上的超高速入水形成的超空泡,精细地观测了水面剧烈喷溅、表面闭合和深闭合的先后发生、以及水中气泡流等一系列有趣流动现象。施红辉等\cite{shi2001measurement, honghui2004underwater}又通过试验测量了钝头体入水的水下声场特性,还分析了细长体头型等因素对入水空泡形态演化和自由面波动特性的影响规律\cite{施红辉2012伴随超空泡产生的高速细长体入水实验研究}。顾建农和张志宏等利用高速摄像研究了球形弹和手枪子弹的入水问题,分析了两种弹丸在不同入水条件下的空泡形态以及水中弹道变化规律。易文俊等\cite{周家胜2007水下射弹的空泡形态特性研究}采用基于RANS的单一介质可变密度混合模型,对水下射弹空泡进行了数值模拟研究。魏卓慧和王树山等\cite{魏卓慧2010experimental}针对球体入水过程的空泡形态和速度变化开展试验研究,分析了不同入水角度的空泡深闭合特性。金大桥\cite{金大桥2010水下动能射弹空泡形态及流体动力特性研究}对细长圆柱体以100m/s以下速度水平入水时的超空泡进行了试验和数值模拟,给出了超空泡形态和流体动力的变化规律。张伟等\cite{张伟2011弹体高速入水特性实验研究, guo2012experimental, guo2012investigation}也研究了多种不同头型细长体的水平高速入水过程,通过研究空泡特性获得了阻力与弹道的关系。曹伟等\cite{曹伟2012自然超空泡航行体弹道稳定性分析}数值模拟研究了航行体以$110m/s$高速入水形成的空化数连续变化的自然超空泡,通过建立简化运动方程分析了航行体的水弹道稳定性。王聪等\cite{王聪2012空气压强对垂直入水空泡影响的数值研究}采用多相流模型和VOF方法数值模拟了锥头圆柱航行体的中等速度入水过程,分析了空气压强对入水空泡的影响规律。魏英杰和王聪\cite{马庆鹏2014锥头圆柱体高速入水空泡深闭合数值模拟研究}又采用VOF方法对锥头航行体高速入水空泡进行流场模拟,研究了空泡深闭合特性并得到了航行体弹道规律。王聪\cite{何春涛2012典型运动体入水过程多相流动特性研究}、张伟\cite{齐亚飞2016弹体高速入水弹道稳定及空泡特性研究}指导的学位论文中采用试验结合数值和理论建模的手段,系统地研究了球体和柱状航行体以$10m/s-500m/s$的速度垂直和倾斜入水的多相空泡流场和弹道稳定性,研究了头型、入水速度和入水角等因素的影响,并率先探索了航行体并行和串联入水的空泡和流场特性。
