% !TEX root = ../main.tex

\begin{digest}
  Water entry refers to the process in which a vehicle enters the water surface from the air. When the vehicle enters the water, it will carry a part of the gas into the water, or it will vaporize the water into water vapor, which is the phenomenon of water entry cavitation/cavitation. The water entry process involves the vehicle and the air-water surface, and the fluid properties of air and water play a major role in the movement of the vehicle. In the process of water entry, the contact between the vehicle and the air-water surface will occur, so the sudden change of force and moment will occur. Considering the turbulent characteristics and complexity of the fluid itself, the whole is a highly nonlinear problem. Motion is carried out in a wave field, which itself is a time-varying system. The problem of water entry is a key issue in the current scientific research field, and the research has a long history.

  At the current stage, cavitation into water is a hot research direction of water entry. And with the continuous enhancement of the scientific and technological strength of the world's military powers, the ocean has increasingly become the main battlefield of modern warfare that countries are competing for. As an important part of modern naval weapons and equipment, the development of underwater weapons such as air-dropped torpedoes, anti-submarine missiles, rocket depth bombs, and high-speed projectiles has always been a hot issue that countries attach great importance to. After the underwater vehicle is launched from the airborne or shipborne platform, it must go through the process of entering the water through the free water surface. The water entry process, as the beginning of the underwater voyage, is a key bottleneck restricting the development of supercavitating weapons. First, in order to facilitate the formation of underwater supercavitation, high-speed water entry should be used, and the impact of high-speed water entry will cause damage to the structure and components of the vehicle; secondly, the entry cavitation has a great impact on the ballistic stability of the vehicle after entry. In short, the water entry process is an uncontrolled stage with the most complex flow phenomena and loads, and the most uncertain attitude and trajectory of the vehicle during the entire navigation process. It also provides initial conditions for the subsequent underwater navigation phase. Controllability and the effectiveness of end-stage attacks are critical. Therefore, the water entry process is the most important link in the launch process, and it is also the key to the success or failure of the launch. The research on the problem of cavitation into water in the wave field environment is relatively insufficient, so this paper studies the problem of cavitation into the water in the wave field environment.
  
    The study of water entry cavitation is a long-standing problem, and has always been a hot and difficult research topic in the field of fluids. As early as the early 20th century, there have been many studies on the impact of water entry, and then to the second half of the 20th century, the problems of water entry cavitation and cavitation gradually attracted research attention. For the water entry problem, the early research methods include theoretical solution, experimental verification, etc., and then gradually began to use finite element, PIV or CFD numerical simulation methods. The water entry process is a very complex nonlinear problem. Although we already have very rich mathematical descriptions of fluid problems, it is almost impossible to solve the specific water entry process purely through theory. An important problem in the early research on the water entry problem is to calculate the maximum impact force on the object during the water entry process through theoretical analysis, but it has been proved to be very difficult. Although experiments are the most reliable way to study the process, in practice it is not practical to evaluate the water pressure on a vehicle experimentally because it would require installing a large number of pressure sensors to cover all sensitive areas on the vehicle. Therefore, measurements can only be made in limited locations, which means that the assessment will not be comprehensive. Alternatively, modeling methods can overcome the limitation of the number of sensors, but it is difficult to accurately predict the water-induced pressure on the vehicle. Take Finite Element Analysis (FEA) as an example, it requires the input of pressure at a specific location on the surface of the vehicle to perform the relevant structural analysis, but FEA itself does not have a reliable source to provide this pressure input. The most widely studied approach to the entry of vehicular bodies into water is the use of grid-based CFD methods, or finite volume methods, which are widely used and well established. CFD has been widely used to predict fluid behavior and fluid-induced structural loads, motion, and deformation. In this paper, the CFD method is used to numerically simulate the water entry cavitation problem.
  
    When a vehicle enters the wave field, the behavior of water and air in both the vehicle and the wave field can be described by mathematical models based on physics and fluid mechanics. Including fluid continuity equation, momentum equation and energy conservation equation. Afterwards, the related changes of the basic equations of fluid mechanics in the multiphase flow environment are discussed. Finally, the water entry cavitation model is discussed. These mathematical models are the basis for performing numerical simulations.
  
    In order to realize the numerical simulation, this paper uses the fluid mechanics calculation software Fluent to carry out the numerical simulation. Firstly, this paper introduces the adopted geometric model and mesh division. Secondly, the boundary conditions of each edge are given, and the specific wave elimination method is discussed. In order to prevent the wave reflection phenomenon caused by the boundary of the water outlet, a part of the wave absorbing area is added to the wave field, and the method of the momentum source term wave absorbing method is used in this area. In order to better capture the interface, the VOF method is used to capture. In the process of entering the water, in order to make the flow field around the vehicle more accurate, the overlapping grid technology is used, and the dynamic mesh surrounding the vehicle and the static mesh of the environment can be integrated to calculate the flow field. Finally, the specific experimental design of this experiment is discussed. According to different incident angles and incident phases, a total of 9 groups of experiments are carried out.
  
    Finally, this paper analyzes and summarizes the phenomena of 9 groups of numerical simulation experiments. First, the stable wave field results are shown, and second, the density distribution at different water entry stages is roughly described. Then, the density field, pressure field and velocity field in the process of water entry are described in detail by taking one example of $60 ^\circ$ entering water and $90 ^\circ$ entering water as examples. Finally, the characteristics of the water entry process of the vehicle in each case are analyzed and discussed. When the vehicle just entered the water surface, cavitation appeared around the vehicle, and there were cavities from the head to the water surface. With the continuous operation of the vehicle, the cavitation on the lower right side gradually collapsed, and the cavitation near the head of the vehicle gradually disappeared, and the cavitation gathered at the tail of the vehicle. The maximum front-end pressure occurs when the object just touches the water surface, and the pressure rises rapidly and reaches the maximum value when the object just touches the water surface. Affected by high pressure, but remains relatively low relative to the first contact with the water, and does not change much during sailing. There is a certain low pressure area on both sides of the vehicle, and the low pressure area occurs in the area where there are cavities. In the part where there are cavities, the fluid moves faster. When the water entry phase is different, the displacement, velocity, angular velocity, and the resultant force and moment of the vehicle will change. The above results are in line with the general law of water entry cavitation, and reflect the influence of different water entry phases and water entry angles on the water entry process.
\end{digest}
